%presentation about Gjonaj_20003

\documentclass[colorbacktitle,inverttitle,landscape,presentation,
	english,
	aspectratio=43, %43 or 169, 1610
	accentcolor=tud9b, %tud9b (temf) or tud5b (gsce) 
]{tudbeamer}

%%additional packages
%%language:
\usepackage[utf8]{inputenc}
\usepackage[english]{babel}
%%math:
\usepackage{amsmath}
\usepackage{mathtools}
%%tikz:
\usepackage{tikz}
\usetikzlibrary{patterns}
\usepackage[siunitx,americaninductors]{circuitikz}
\usepackage{siunitx}
%%pgfplots:
\usepackage{pgfplots}
\usepgfplotslibrary{groupplots}
\usepgfplotslibrary{patchplots}
\usepackage{pgfplotstable}
\pgfplotsset{compat=newest}\usepgfplotslibrary{units}
%%scaling:
\usepackage{adjustbox}
%%temf:
\usepackage{temf-finite}
\usepackage{temf-fields}
\usepackage{temf-matrixvector}
\usepackage{notInTemfMacros}

%needs to come last
\usepackage{temfbeamer}

%remove TU logo from every slide but the title
\setbeamertemplate{headline}[TUD theme nologo] 

%set date
\date{May 30, 2018}

%%%%%%%%%%%%%%%%%%%%%%%%%%%%%%%%%%%%%%%%%%%%%%%%%%%%%%%%%%%%%%%%%%%%%%%%%%%%%%%%%%%%%%%%%%%%%%%%%%%%%%%%%%%%%%%%%%%%%%%%%%%%%%%%%%%%%%

\title{Conformal Modeling of Space-Charge-Limited Emission from Curved Boundaries in Particle Simulations}

\subtitle{\\[0.3\baselineskip]
	Peter Förster\\[0.3\baselineskip]
{\small Jacopo Corno, M.Sc. and Prof. Dr. Sebastian Schöps}\\
[0.3\baselineskip]
{\tiny Proseminar ETiT}\\[0.3em]
	\mbox{\scriptsize}~}
	
\institute[TU Darmstadt | Fachbereich 18 | Institut Theorie Elektromagnetischer Felder]{Institut für Theorie Elektromagnetischer Felder, TU Darmstadt}

%%%%%%%%%%%%%%%%%%%%%%%%%%%%%%%%%%%%%%%%%%%%%%%%%%%%%%%%%%%%%%%%%%%%%%%%%%%%%%%%%%%%%%%%%%%%%%%%%%%%%%%%%%%%%%%%%%%%%%%%%%%%%%%%%%%%%%

\begin{document}
	
\begin{titleframe}
	\tudtitle[images/temf_logo.pdf]{images/temf_background.jpg} 
	%\tudtitle[images/gsce_logo.pdf]{images/gsce_background.jpg}
	\end{titleframe}
	
\begin{frame}
	\frametitle{Outline}
	\tableofcontents%[currentsection,subsectionstyle=show/show/hide]
\end{frame}
	
%%%%%%%%%%%%%%%%%%%%%%%%%%%%%%%%%%%%%%%%%%%%%%%%%%%%%%%%%%%%%%%%%%%%%%%%%%%%%%%%%%%%%%%%%%%%%%%%%%%%%%%%%%%%%%%%%%%%%%%%%%%%%%%%%%%%%%
	
\section{Motivation}
\begin{frame}
\frametitle{Motivation}
	\begin{itemize}
	\item electromagnetic Particle-In-Cell simulations are used to investigate particle physics in accelerators (e.g. FD, FDTD, FIT on orthogonal grids)
	
	\item modeling curved material boundaries on such grids presents difficulties
	
	\item low accuracy of boundary fields at curved emission surfaces greatly affects overall performance
	
	\item boundary-conformal approach retains advantages and provides accurate solutions for curved emission boundaries
	\end{itemize}
\end{frame}
		
%%%%%%%%%%%%%%%%%%%%%%%%%%%%%%%%%%%%%%%%%%%%%%%%%%%%%%%%%%%%%%%%%%%%%%%%%%%%%%%%%%%%%%%%%%%%%%%%%%%%%%%%%%%%%%%%%%%%%%%%%%%%%%%%%%%%%%

\section{Problem Setting}
\begin{frame}
\frametitle{Mathematical Model}
	\begin{itemize}
	\item the model equations read
	\begin{align*}
	\frac{\d\rmat_i}{\d t} &= \vmat_i\\
	\frac{\d m_e\vmat_i}{\d t} &= e(\Emat + \vmat_i\times\Bmat),\quad i = 1\dots N
	\end{align*}
	where $e$ is the electron charge, $m_e$ is the relativistic electron mass; $\rmat_i$ and $\vmat_i$ denote particle positions and velocities respectively
	\end{itemize}
\end{frame}

\begin{frame}
\frametitle{Mathematical Model}
	\begin{itemize}	
	\item only taking into account electrostatic particle-particle interactions the space-charge field $\Emat$ is determinded by
	\begin{align*}
	\nabla(\eps\Emat) &= \frac{1}{\eps_0}\sum_i^N q_i \delta(\rmat-\rmat_i)\\
	\Emat &= -\nabla \phifield,
	\end{align*}
	where $\eps$ is the permittivity and $q_i$ the charge of the $i$-th particle	
	\end{itemize}
\end{frame}

\begin{frame}
\frametitle{Space Charge Limited Emission}
	\begin{itemize}
	\item the Child-Langmuir diode equation reads
	\begin{align*}
	J_\mathrm{CL} = \left(\frac{4 \eps_0}{9}\right) \sqrt{\frac{2 e}{m_e}} \frac{\delta \phifield_b^{3/2}}{\delta d^2},
	\end{align*}
	where $J_\mathrm{CL}$ is the current at the emission surface and $\delta\phifield_b$ the local potential difference at a distance $\delta d$ from the surface
	\end{itemize}		
\end{frame}

%%%%%%%%%%%%%%%%%%%%%%%%%%%%%%%%%%%%%%%%%%%%%%%%%%%%%%%%%%%%%%%%%%%%%%%%%%%%%%%%%%%%%%%%%%%%%%%%%%%%%%%%%%%%%%%%%%%%%%%%%%%%%%%%%%%%%%

\section{Conformal Method}
\begin{frame}
\frametitle{Discrete Field Equations}	
	\begin{itemize}
	\item using the Finite Integration Technique the discrete formulation of the equations reads
	\begin{align*}
	\divfit \dfit &= \qfit\\
	\dfit_i &= \iint_{\Delta A_i} \eps(\nabla \phifield)\dA
	\end{align*}
	
	\item $\dfit$ denotes the vector of electrostatic fluxes, $\divfit$ the discrete $\div$-operator, $\qfit$ the total charge contained in each dual cell and $\Delta A_i$ the area element of the $i$-th dual cell face
	
	\item discretization error stems only from approximating the flux integrals $\dfit_i$ using potential grid values $\phi_i$
	
	\item conformal approach takes into account the geometry of the boundary and minimizes error
	\end{itemize}
\end{frame}

\begin{frame}
\frametitle{Particle Injection}
	\begin{itemize}
	\item emitted particles are assigned to predetermined positions
	
	\item local currents associated to the individual particles are calculated using Child's law
	
	\item consistency between the currents and the field solution is enforced through iteration until constant emission current is established
	\end{itemize}
\end{frame}

%%%%%%%%%%%%%%%%%%%%%%%%%%%%%%%%%%%%%%%%%%%%%%%%%%%%%%%%%%%%%%%%%%%%%%%%%%%%%%%%%%%%%%%%%%%%%%%%%%%%%%%%%%%%%%%%%%%%%%%%%%%%%%%%%%%%%%

\section{Results}
\begin{frame}
\frametitle{Results}
	\begin{itemize}
	\item steady state emission current is observed on the source surface for different mesh resolutions
	
	\item using a staircase approximation 1.5 mio. mesh nodes are needed for convergence
	
	\item staircase model also introduces numerical oscillations caused by low order field approximation at the emission surface
	
	\item the relative error for fine to moderate mesh sizes varies between 10-25\%
	
	\item using the conformal method only 130k nodes are needed for convergence
	
	\item computational effort can be reduced since grid-field-equations contain fewer nodes
	\end{itemize}
\end{frame}

%%%%%%%%%%%%%%%%%%%%%%%%%%%%%%%%%%%%%%%%%%%%%%%%%%%%%%%%%%%%%%%%%%%%%%%%%%%%%%%%%%%%%%%%%%%%%%%%%%%%%%%%%%%%%%%%%%%%%%%%%%%%%%%%%%%%%%

\begin{frame}
\frametitle{Conclusion}
	\begin{itemize}
	\item boundary conformal discretization of electrostatic field equations and boundary conformal particle injection technique
	
	\item method enforces field-space-charge consistency from Child's law
	
	\item solution of a Pierce gun model shows advantages over staircase approximation
	\end{itemize}
\end{frame}

%%%%%%%%%%%%%%%%%%%%%%%%%%%%%%%%%%%%%%%%%%%%%%%%%%%%%%%%%%%%%%%%%%%%%%%%%%%%%%%%%%%%%%%%%%%%%%%%%%%%%%%%%%%%%%%%%%%%%%%%%%%%%%%%%%%%%%

\begin{frame}
\frametitle{List of References}
	\begin{enumerate}
	\item[{[1]}]
E. Gjonaj, T. Lau and T. Weiland. Conformal Modeling of Space-Charge-Limited Emission from Curved Boundaries in Particle Simulations. In Proceedings of the 2003 Particle Accelerator Conference May 12-16, 2003 in Portland, OR, USA.

	\item[{[2]}]
B. Krietenstein, R. Schuhmann, P. Thoma and T. Weiland. The Perfect Boundary Approximation Technique Facing the Big Challenge of High Precision Field Computation. In Proceedings of the XIX International Linear Accelerator Conference, 1998.
	\end{enumerate}
\end{frame}
	
	
\end{document}